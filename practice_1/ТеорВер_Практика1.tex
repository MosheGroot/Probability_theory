\documentclass{article}
\usepackage[T1,T2A]{fontenc}
\usepackage[utf8]{inputenc}
\usepackage[english,russian]{babel}

\usepackage[left=3cm,right=3cm,
    top=3cm,bottom=3cm,bindingoffset=0cm]{geometry}

\usepackage{graphicx}
\usepackage{color}
\usepackage{hyperref}

\usepackage{setspace}
\usepackage{indentfirst}
\usepackage{textcomp}
\usepackage{ifthen}
\usepackage{calc}

\title{Теор. Вер.}
\author{Практическое занятие 1}

\begin{document}
\maketitle
\section{Сочетание}

Сочетание -- подмножество, содержащее k элементов, выбранном в множестве, содержащее n элементов (k < n)
Таким образом, сочетания различаются по составу, но не по порядку элементов.

Допустим, имеем подмножество:

$\{a, b, c\}$

Тогда мы можем взять следующие сочетания размеров $k$:

для $k = 0$: $\emptyset$

для $k = 1$: $\{a\}, \{b\}, \{c\}$

для $k = 2$: $\{a, b\}, \{a, c\}, \{b, c\}$

для $k = 3$: $\{a, b, c\}$

С использованием обозначений количества сочетаний, имеем:

\quad$C_3^0 = 0$ \quad $C_3^1 = 3$ \quad $C_3^2 = 3$ \quad $C_3^3 = 1$

\section{Перестановка}

Перестановка -- упорядоченный набор элементов множества.
То есть перестановки одного множества не отличаются по составу, но отличаются по порядку.

Допустим, имеем то же подмножество:

$\{a, b, c\}$

Тогда можем построить следующие перестановки:

$<a, b, c>$

$<a, c, b>$

$<b, a, c>$

$<b, c, a>$

$<c, a, b>$

$<c, b, a>$

И с использованием обозначений можем вычислить количество:

$$P_3 = 3 * 2 * 1 = 3! = 6$$

\section{Размещение}

Размещение -- это подмножество из $k$ элементов, взятое в множестве из $n$ элементов в  определенном порядке.
Таким образом, размещения различаются либо порядком элементов, либо составом элементов, либо и порядком, и составом

Возьмем вновь $\{a, b, c\}$.

Тогда для $k = 0$: $\emptyset$

для $k = 1$: $<a>, <b>, <c>$

для $k = 2$: $<ab>, <ba>$

\qquad\qquad\quad$<ac> <ca>$

\qquad\qquad\quad$<bc> <cb>$

для $k = 3$: $<abc> <acb>$

\qquad\qquad\quad$<bac> <bca>$

\qquad\qquad\quad$<cab> <cba>$
\\
И, вводя обозначения, получаем:

$A^0_3 = 1$, \quad$A^1_3 = 3$,\quad $A^2_3 = 6$,\quad $A^3_3 = 6$


\section{Формулы для сочетания, перестановок и размещения}

$$C^k_n = \frac{n!}{(n - k)!k!}, 0 <= k <= n$$

$$P_n = n!$$

$$A^k_n = \frac{n!}{(n - k)!}, 0 <= k <= n$$

Также можно отметить, что количество способов упорядочивания пустого множества равно $0! = 1$.

\section{Бином Ньютона}

Сама формула Бинома Ньютона:
$(a + b)^n = \sum{C^k_na^kb^{n-k}} $

И так же с ней можно связать так называемый "Треугольник Паскаля" ($ n = 1...5$):

$$1$$
$$1 \quad 1$$
$$1 \quad 2 \quad 1$$
$$1 \quad 3 \quad 3 \quad 1$$
$$1 \quad 4 \quad 6 \quad 4 \quad 1$$

При этом, можно отметить, что биномиальные коэффициенты являются семмитричными, то есть: $C_n^k = C_n^{n - k}$

Также мы можем организовать формулу разложения суммы:

$$\sum_{k = 0}^{n}{C_n^k} = \sum_{k = 0}^{n}{C_n^k * 1^k * 1^{n - k}} = (1 + 1)^n = 2^n$$

\section{Примеры}
\subsection{Пример 1}

Имеем множество цифр: $\{0, 1, 2, 3, 4\}$

Вопрос: какое колво пятизначный чисел можно записать этими числами без повторений?

Можно отметить, что в качестве первой цифры нельзя использовать $0$.
То есть в итоге получаем: $4 * 4 * 3 * 2 * 1 = 96$

\subsection{Пример 2}

Имеем те же цифры $\{0, 1, 2, 3, 4\}$

Вопрос: сколько четных (!) пятизначных чисел можно записать из этих цифр?

Потребуется рассмотреть два варианта: отдельно числа с $0$ на конце, отдельно числа с $2, 4$ на конце. Отдельно рассматривать ноль на последнем месте приходится, так как ноль не должен стоять на первом месте в вариантах с $2, 4$ на конце.

Для чисел с $0$ на конце получим следующее количество:
$$4 * 3 * 2 * 1 * 1$$

Для чисел с $2$ или $4$:
$$3 * 3 * 2 * 1 * 2$$

Полученное количество достаточно просуммировать, и получим ответ.

\subsection{Пример 3}

Задача: 7 девушек водят хоровод, сколькими способами они могут встать в круг?

Решение: подсчитать перестановки? Однако, важно не учитывать сдвиги, так как мы хоровод подразумевает круг!

В круге нет первого места, значит мы его можем построить относительно кого-то. Выберем одного участника, и расположем относительно него всех остальных, то есть 6 человек. Итого получаем перестановку из 6 "элементов": 

$$P_6 = 6! = 720$$

\subsection{Пример 4}

Задача: имеем 7 различных бусинок, сколькими способами можно собрать ожерелье из всех бусинок?

так же задача? Нет. В отличии от хоровода, мы можем перевернуть "с ног на голову" ожерелье, и само ожерелье будет с тем же порядком бусинок (симметрия). То есть количество будет меньше в два раза:

$$\frac{P_6}{2} = 360$$

\subsection{Пример 5}

! (решение под вопросом, записал что было) !

Имеем: в коробке $6$ белых шаров и $4$ черных.

Задача: берем из коробки $4$ шара. Сколькими способами можно достать $3$ белых и $1$ черный?

$C_6^3 * C_4^1 = \frac{6!}{3!(6 - 3)!} = \frac{6 * 5 * 4 * 3 * 2 * 1}{3 * 2 * 1 * 3 * 2 * 1} = \frac{6 * 5 * 4}{6} * 4 = 80$

Так как:

$C_n^0 = C_n^n = 1$

$C_n^1 = C_n^{n - 1} = n $

$C_n^2 = C_n^{n - 2} = \frac{n(n - 1)}{2} $

$C_n^3 = C_n^{n - 3} = \frac{n(n - 1)(n - 2)}{2} $


\subsection{Пример 6}

Имеем: колода карт, 36 листов.
Вопрос: сколькими способами мы можем вытащить четыре карты, среди которых две дамы, причем одна из них дама пик.

Решение. В колоде 36 часть, разделим ее на части: "дама пик", "3 других дамы", "остальные 32". Из нее мы хотим взять набор: "дама пик", "1 дама", "2 других карты".

\begin{tabbing}
дама пик             \quad\qquad\qquad\qquad $\rightarrow$ \qquad дама пик \\
3 других дамы        \qquad\qquad $\rightarrow$ \qquad 1 дама \\
остальные 32 карты   \qquad $\rightarrow$ \qquad 2 карты \\
\end{tabbing}

Тогда, получаем выбор:

$$ C_1^1 * C_3^1 * C_{32}^2 = 1 * 3 * \frac{32 * 31}{2} = 1488$$

\subsection{Пример 7}

Имеем: в буфете 4 сорта пироженных.
Требуется: выбрать 7 штук пироженных, сколькими способами это можно сделать?

По сути, это сочетание с повторениями.

Обозначим пироженные, например, буквами \{a,b,c,d\}. Попробуем взять такой набор: $\{a, b, c, d * 4\}$

Или такой: $\{a * 2, b * 0, c * 2, d * 3\}$

И много других вариантов. Но надо понять, что определяет этот выбор. По сути, мы можем так же сделать следующие интерпретации наших вариантов:

\begin{center}
\begin{tabbing}
\qquad\qquad\qquad\qquad[\quad*\quad|\quad*\quad|\quad*\quad|\quad****\quad]\\

\qquad\qquad\qquad\qquad[\quad**\quad|\qquad|\quad**\quad|\quad***\quad]
\end{tabbing}
\end{center}

Границы с квадратными скобками мы передвигать не можем. Так же имеем 3 внутренних стенки, которые можем передвигать как угодно (в пределах внешних границ). Получается, колво всего элементов -- 10 (7 пироженных и 3 стены). То есть в итоге, у нас получается следующее сочетание:

$$C_{10}^3 = C_{10}^7$$

$$C_{10}^3 = \frac{10 * 9 * 8}{6} = 120$$

Ответ: $120$

\subsection{Пример 8 (самостоятельно)}

Сколькими способами можно разместить 7 белых шаров в 4 урнах?

По сути формулировка предыдущей задачи, если будет интересно порешать (не обязательно).

\subsection{Пример 9 (самостоятельно)}

Сколько неотрицательных целочисленных решений имеет уравнение:

$$x_1 + x_2 + x_3 + x_4 = 7$$

Так же доп. формулировка задачи с пирожеными, если хочется попрактиковаться.

\subsection{Отступление про проверки}

Возможные проверки при заданиях:\\
$+$ -- правильное решение\\
$+_{.}$ -- правильное решение с помарками\\
$\pm$ -- решение с некритическими ошибками\\
$\mp$ -- решение с ошибками\\
$-_{.}$ -- серьезные ошибки\\
$-$ -- неправильное решение\\
$0$ -- полный ноль\\

\section{Возведение в степень суммы $k$ слогаемых}

Помимо бинома Ньютона, нам очень может понадобиться такое действие, как возведение в степень $k$ слогаемых:

$$(a_1 + a_2 + ... + a_k) ^ n =$$
$$ = \sum_{n_1 + ... + n_k = n}{\frac{n!}{n_1!n_2!...n_k!} * a_1^{n_1}* a_2^{n_2} *... * a_k^{n_k}}$$

Так называемая полиномиальная формула.

\subsection{Пример}

Имеем отрезовк $[0, 10]$.

Задача: поставить $5$ точек так, чтобы 2 точки $\in (0, 1)$  и 3 точки $\in (8, 10)$.
При этом пусть у каждой точки будет свой номер.


Иная формулировка: 5 шаров, 3 корзины. Сколькими способами можно распределить мячи так, чтобы 2 шара было в первой корзине, 3 в третьей, а центральная -- пустая?

Ответ:

$$\frac{5!}{2! * 0! * 3!} = \frac{4 * 5}{2} = 10 $$


\end{document}
