\documentclass{article}
\usepackage[T1,T2A]{fontenc}
\usepackage[utf8]{inputenc}
\usepackage[english,russian]{babel}

\usepackage[left=3cm,right=3cm,
    top=3cm,bottom=3cm,bindingoffset=0cm]{geometry}

\usepackage{graphicx}
\usepackage{color}
\usepackage{hyperref}
\usepackage{amsmath}
\usepackage{amsfonts}

\usepackage{setspace}
\usepackage{indentfirst}
\usepackage{textcomp}
\usepackage{ifthen}
\usepackage{calc}

\title{Теория Вероятностей и Математическая Статистика\\
ФИИТ, 2 курс, 4 семестр}
\author{Лекция 4}

\begin{document}
\maketitle

\section{Случайное явление. Парадокс Бертрана}

Для того, чтобы иметь представление о случайном явлении, рассмотрим пример -- \textbf{парадокс Бертрана}:
\\


Имеется круг, в который вписан равносторонний треугольник. В круге наугад выбирается хорда. Нужно найти вероятность того, что эта хорда окажется больше стороны треугольника.

\begin{center}
    \includegraphics[scale=0.4]{1.png}
\end{center}

\begin{enumerate}
\item В первую очередь рассмотрим выбор хорды наугад.

Зафиксируем вершину треушольника в виде точки и выберем хорду, выходящую из данной вершины (для этого выберем точку на окружности, и соединим вершину с этой точкой). Мы можем провести две различных хорды, при этом длина первой хорды (до точки 1) $N1 < $ стороны треугольника. Длина второй $N2 > $ стороны треугольника.

Таким образом, только на заштрихованной дуге у нас окажется хорда длиннее стороны треугольника. Соответственно, так как треугольник равносторонний, эта дуга занимает $\frac{1}{3}$, и итоговая вероятность $P = \frac{1}{3}$.

\begin{center}
    \includegraphics[scale=0.4]{2.png}
\end{center}

\item Посмотрим на эту задачу немного под другим углом. Построим дополнительный диаметр. Наша хорда -- это линия, которую мы построим перпендикулярно диаметру (соответственно, пересекается с ним в некоторой точке).

В каком случае наша хорда будет оказываться больше стороны треугольника? По рисунку видно, что вторая хорда точно больше стороны треугольника. И в целом, мы можем выделить серединную область, где длина хорд будет больше длины стороны треугольника -- ниже стороны треугольника длина хорды будет, очевидно, меньше, а сверху область будет ограничена симметрично нижней, -- таким образом она займет половину круга, то есть вероятность $P = \frac{1}{2}$.

\begin{center}
    \includegraphics[scale=0.4]{3.png}
\end{center}

\item Рассмотрим еще один вариант! будем выбирать хорду внутри нашей области и построим хорду, для которой эта точка будет служить центром. Требуется указать такие точки, когда хорда будет длиннее стороны треугольника.

Для этого в треугольник впишем еще одну окружность. Если провести хорду, которая касается сверху этой окружности, то она окажется такой же, как и нижняя сторона треугольника. Если хорда сдвинется вовнутрь, то она станет больше стороны.

То есть все нужные нам точки находятся внутри этого вписанного круга (закрашено на рисунке). Соответственно, отношение площадей кругов даст нам нужную вероятность: $P = \frac{1}{4}$

\begin{center}
    \includegraphics[scale=0.4]{4.png}
\end{center}

\end{enumerate}

\quad

Итого мы имеем целых 3 варианта. Интересно то, что все они правильные. Ключевым словом является \textbf{наугад}. И вариант выбора наугад (или в принципе любого случайного или непредсказуемого явления), то есть сопоставление этому слову конкретной математической модели, определяет итоговое решение. То есть если детермизированное описание этого явления отсутствует, то, соответственно, и варианты ответа могут оказаться разные. В этом и состоит сложность этого парадокса.

\section{Случайные величины}

Эта тема -- второй раздел тем нашего семестра. Мы будем рассматривать, каким образам нашим случайным величинам можно сопоставить в соответствие некоторые числа.
\\

Введем обозначения. Случайные величины будем обозначать греческими буквами: $\xi$ (кси); $\eta $ (эта), $\zeta $ (дзета).

Большие латинские буквы $A, B, C$ -- по прежнему обозначения для событий  .

Рукописные буквы $\mathcal{ABC}$ -- классы множеств. Нас будет интересовать $\mathcal{A}$ -- $\sigma$-алгебра  .

$F, f, g$ -- функции.

$\tau, t, x, y, z$ -- переменные.
\\

Мы будем работать в вероятностном пространстве $(\Omega, \mathcal{A}, P)$. В этом вероятностном пространстве мы построим еще одну характеристику.

\subsection{Случайная величина}

\textbf{Определение.} Случайная величина $\xi$ заданная в вероятностном пространстве $(\Omega, \mathcal{A}, P)$ представляет собой функцию:

$$ \xi : \Omega \rightarrow \mathbb{R}^1,$$

отображающую множество исходов опыта (не обязательно элементарных) на числовую прямую так, что множество исходов, для которых случайная величина меньше $x$ является элементом $\sigma$-алгебры:

$$\{\omega : \xi(\omega) < x, x \in \mathbb{R}^ 1, y \in \mathcal{A}\}$$


Как и случайные события, случайная величина задается достаточно непросто. И для того, чтобы с этим дальше работать, нам потребуется еще одно определение.

\subsection{Функция распределения}

\textbf{Определение}. Функцией распределения случайной величины $\xi$: 

$$F_{\xi}(X) = F(x) = P(\{\omega : \xi(\omega) < x\}), $$

представляет собой вероятность такого множества, для которых величина $\xi$ принимает значения меньше $x$ для любых действительных $x$. Также можно записать и более кратко данное определение:

$$ = P(\xi < x) $$

То есть речь идет о тех элементах множества исходов, для которых величина $\xi$ приняла значение меньше $x$. Поскольку это множество в соответствии с определением нашей случайной величины -- элемент $\sigma$-алгебры, ему всегда ставится в соответствие некоторая вероятность, а значит эта вероятность существует всегда и может быть найдена.

\textbf{Замечание:} можно отметить, что функция распределения существует всегда и есть у абсолютно любой случайной величины.

\subsubsection{Пример}

Три монеты. Мы знаем, что элементарные исходы будут следующие:

\begin{center}
0 0 0\\
0 0 1\\
0 1 0\\
0 1 1\\
1 0 0\\
1 0 1\\
1 1 0\\
1 1 1
\end{center}

Пусть $\xi$ -- кол-во гербов. Тогда различным исходам будет соответствовать различное колво гербов:

\begin{center}
    \includegraphics[scale=0.7]{8.png}
\end{center}

Постром функцию распределения $F(x)$. Рассмотрим конкретные точки:
\\

$F(-1) = P(\xi < -1)$ -- невозможно, так как множество значений у нас дискретное и конечное: $\{0, 1, 2, 3\} $.

$F(0) = P(\xi < 0)$ -- опять же, меньше нуля наша случайная величина быть не может.

$F(0.1) = P(\xi < 0.1) = P(\xi = 0) = \frac{1}{8}$

$F(0.7) = P(\xi < 0.7) = P(\xi = 0) = \frac{1}{8}$
\\

Из двух последних значений можно заметить, что нас интересуют точки изменения значений (дробей как выше мы можем подобрать сколько угодно для одного значения, на то у нас и функция).
\\

$F(1) = P(\xi < 1) = P(\xi = 0) = \frac{1}{8}$

$F(2) = P(\xi < 2) = P(\xi = 0) + P(\xi = 1) = \frac{4}{8}$

$F(3) = P(\xi < 3) = P(\xi = 0) + P(\xi = 1) + P(\xi = 2) = \frac{7}{8}$

$F(4) = P(\xi < 4) = P(\xi = 0) + P(\xi = 1) + P(\xi = 2) + P(\xi = 3) = 1$
\\

Таким образом, функция оказалась дискретной, кусочно-постоянной:

\begin{center}
    \includegraphics[scale=0.6]{5.png}
\end{center}

Стрелкам подчеркивается, что в конкретной точке достигается только граница с левой стороны, граница с правой не достигается.

\subsection{Классификация случайных величин}

Дальше нас интересует классификация случайных величин. Зависить случайная величина будет от типа функции распределения.\\

Случайная величина называется \textbf{дискретной случайной величиной} (ДСВ), если функция распределения СВ $F$ -- кусочно-постоянная.
\\

Случайная величина называется \textbf{абсолютно непрерывной (непрерывной) случайной величиной} (НСВ), если $F$ -- диффиренцируема почти во всех точках (за исключением конечного количества). 
\\

\textbf{Сингулярные случайные вличины} (ССВ) -- случайные величины, функции распределения которых не диффиренцируемы и к кусочно-постоянным функциям не относятся.

\begin{center}
    \includegraphics[scale=0.6]{6.png}
\end{center}


\subsection{ДСВ}

Величина $p_k = F(X_k + 0) - F(x_k)$ -- скачок функции распределения в точке разрыва.

Для предыдущего примера:

\begin{center}
    \includegraphics[scale=0.6]{7.png}
\end{center}

То есть вероятность $P(\xi = x_k) = p_k = F(X_k + 0) - F(x_k)$.
\\

Таким образом, дискретная случайная величина сосредоточенна в точках разрыва.

\subsection{НСВ}

Функция распределения допускает представления в виде интеграла с переменным верхним пределом от $f(x)$:

$$ F(x) = \int\limits_{-\infty}^{x} f(t)dt$$

Тогда $f(x) = \frac{dF(x)}{dx}$, то есть производная, и называется она -- \textbf{плотность вероятности}.

\subsection{Свойства функции распределения случайной величины}

\begin{enumerate}
\item Ограниченность

\qquad$0 \leq F(x) \leq 1$

\item Неубывающая

\qquad$x_1 < x_2 \Leftrightarrow F(x_1) \leq F(x_2)$

\item Непрерывная слева

\qquad$\lim\limits_{x \rightarrow x_0 - 0} F(x) = F(x_0)$

\item Поведение на бесконечности

\qquad$ \lim\limits_{x \rightarrow +\infty} F(x) = 1$

\qquad$ \lim\limits_{x \rightarrow -\infty} F(x) = 0$

\end{enumerate}


\textbf{Замечание:} свойства 1-4 являются характеристическими. Любая функция, обладающая данными свойствами, является функцией рраспределения какой-либо случайной величины.

\subsection{Вероятность случайной величины}

Найдем вероятность попадания случайной величины в полуинтервал:

$$P(\xi \in [a; b)) = P(a \leq \xi < b)$$


Для этого возьмем числовую прямую и представим ее в виде частей:

$$(-\infty; +\infty) = (-\infty; a)\cup[a; b)\cup[b; +\infty)$$

Из свойства 4) $\Rightarrow P(\xi \in (-\infty; +\infty)) = 1$
\\

Учитывая, что части, которые мы взяли, не пересекаются, то вероятность попадания в них это сумма вероятностей, так что можем сделать следующее разложение:

$$P(\xi \in (-\infty; a)) + P(\xi \in [a; b)) + P(\xi \in [b; +\infty)) = 1$$

Перепишем в виде неравенств:

$$ P(a \leq \xi \leq b) = 1 - P(\xi < a) - P(\xi \geq b)$$

Первое слогаемое -- значение распределение в точке $a$, то есть $F(a)$, а во втором слогаемом перейдем к событию противоположному: $1 - P(\xi < b)$. Тогда получим:

$$ = 1 - F(a) - (1 - P(\xi < b)) = 1 - F(a) - 1 + F(b) = F(b) - F(a)$$

Таким образом, результат:

$$ P(a \leq \xi \leq b) = F(b) - F(a)$$
\\


Если выберем конкретную точку $c$, то можно представить, что мы попадаем в некоторый небольшой интервал: $P(\xi = c) = P(c \leq \xi < c + 0) = F(c + 0) - F(c)$:

\quadЕсли $F$ -- непрерывная функция, то $P(\xi = c) = 0$

\quadЕсли $F$ -- разрывна, а $c$ -- точка разрыва, то $P(\xi = c) \not= 0$ -- величина скачка в точке разрыва.
\\

Наша готовая формула очень похожа на формулу Ньютона-Лейбница. Так что:
если существует $f(x)$, то $p(a \leq_{(<)} \xi \leq_{(<)} b) = F(b) - F(a) = \int\limits_a^b f(x)dx$. То есть интеграл от плотности вероятности.

\subsection{Плотность вероятности}

Таким образом плотность вероятности обладает следующими свойствами:

\begin{enumerate}
\item Неотрицательность

\qquad$ f(x) \geq 0$

\item Нормированность

\qquad$ \int\limits_{-\infty}^{+\infty}f(x)dx = 1$

\end{enumerate}

\subsection{Числовые характеристики случайной величины}

Несмотря на то, что функция распределения существует всегда, работать с такими характеристическими функциями достаточно непросто, поскольку характеристика функциональная. Нам хочется исследовать возможность описание случайной величины через числовые характеристики. Конечно, полностью это не удастся, но в значительной степени это получится.

\subsubsection{Математическое ожидание}

Одной из таких характеристик является -- \textbf{среднее значение} СВ (математическое ожидание):

$$M[\xi] = M\xi = \int\limits_{\Omega}\xi(\omega)P(d\omega)$$

Запись $M[\xi]$ -- применение некоторой операции (в данном случае усреднения) к данной величине. Впрочем, мы чаще будем использовать более короткий вариант -- $M\xi$.

Математическое ожидание выражается через интеграл Лебега. Так как интеграл Римана не дает нам работать с разрывными функциями, мы будем пользоваться интегралом Лебега, в котором эти проблемы устранены.

Мы с его особенностями сталкиваться не будем, встречается интеграл только в определении, но все же полезно знать. В задачах мы чаще будем решать пользуясь интегралом Римана или его расширенным вариантов -- интегралом Римана-Стилтьеса.
\\

В случае ДСВ или НСВ математическое ожидание можно вычислить, пользуясь интегралом Римана-Стилтьеса (приводит к двум вариантам записи):

$$M\xi = \int\limits_{\R}xdF(x)$$

\qquad
\begin{cases}
   $= \sum\limits_{k = 1}^{\infty}x_k p_k, p_k = F(x_k + 0) - F(x_k),$ &\text{для ДСВ}\\
   \\
   $= \int\limits_{-\infty}^{+\infty}xf(x)dx,$ &\text{для НСВ}
\end{cases}
\\

Почему мат. ожидание не характеризует случайную величину полностью? Если рассматривать распределение точкек на отрезке $[-1, 1]$, то среднее значение, ожидаемо, $= 0$. Если мы рассмотрим ту же ситуацию на отрезке $[-100; 100]$ среднее значение тоже $0$, однако очевидно, что разброс значений намного больше.

\begin{center}
    \includegraphics[scale=0.6]{9.png}
\end{center}

Таким образом, мы хотим, чтобы одна из числовых зарактеристик должна была бы говорить, как далеко отклоняются возможные случайные величины от ее среднего значения.

Понятно, что величина принимает значения непредсказуемые зарание, а значит и отклонение от среднего значения тоже непредсказуемы заранее. Опять же, из примера выше, эти отклонения могут иметь различную величину, так что дополнительно придется говорить об усреднении этого отклонения.

\subsubsection{Дисперсия}

Средний квадрат отклонения от среднего значения случайной величины -- \textbf{дисперсия} случайной величины.

Величину берем квадратичную, так как это позволит нам избавиться от знака.

$$M(\xi - M\xi)^2 = D[\xi] = D\xi$$

То, что внутри скобок -- отклонение случайной величины. После возведения в квадрат -- квадратичное отклонение случайной величины. А взятие среднего значения -- среднее квадратичное отклонение от среднего значения случайной величины.

\begin{center}
    \includegraphics[scale=0.4]{10.png}
\end{center}

В случае ДСВ или НСВ дисперсия, это интеграл Римана-Стилтьеса:

$$D\xi = \int\limits_{\mathbb{R}} (x - M\xi)^2dF(x) = $$

\qquad
\begin{cases}
$= \sum\limits_{k = 1}^{\infty}(x_k - M\xi)^2 p_k,$ & \text{для ДСВ}\\
\\
$= \int\limits_{-\infty}{+\infty} = (x - M\xi)^2f(x)dx,$ & \text{для НСВ}\\
\end{cases}

\subsubsection{Свойства математического ожидания}

Математическое ожидание -- интеграл, а значит обладает всеми свойствами интеграла.

\begin{enumerate}

\item Математическое ожидание линейной неоднородной комбинации может быть записано следующим образом:

\qquad$M(a\xi + b) = a\cdot M\xi + b$

\item Математическое ожидание для абсолютных величин связяно следующим неравенством

\qquad$(M\xi) \leq M|\xi|$
\end{enumerate}

\subsubsection{Свойства дисперсии}

\begin{enumerate}
\item Линейная комбинация

$$ D(a\xi + b) = M\Bigl[(a\xi + b) - M(a\xi + b)\Bigr]^2 = M\Bigl[a(\xi - M\xi)\Bigr]^2 = a^2 \cdot D\xi$$

\item Дисперсия -- величина неотрицательная

$$D\xi \geq 0 \qquad(D\xi = 0 \Leftrightarrow P(\xi = c) = 1)$$

\item Дисперсия -- может быть выражена как разность между средним значением квадрата и квадрата среднего значения (доказать можно самостоятельно)

$$ D\xi = M(\xi - M\xi)^2 = M\xi^2 - (M\xi)^2$$

\end{enumerate}

Также можно отметить, что размерность дисперсии -- квадратные единицы (так как мы усредняли квадратные единицы), в отличии от математического ожидания. А значит сравнивать со средним значением так просто не получится (а хотелось бы).
\\

\textbf{Определение.} Среднее квадратическое отклонение $\sigma = \sqrt{D\xi}$ или $D\xi = \sigma^2 \geq 0$

\subsection{Дополнение для дз}

Ряд распределения -- это таблица следующего вида:

\begin{center}
$\xi \sim$
\begin{pmatrix}
x_{1} & x_{2} & \ldots & x_{n}& \ldots\\
p_{1} & p_{2} & \ldots & p_{n}& \ldots\\
\end{pmatrix}
\end{center}


\end{document}
