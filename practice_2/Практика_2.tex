\documentclass{article}
\usepackage[T1,T2A]{fontenc}
\usepackage[utf8]{inputenc}
\usepackage[english,russian]{babel}

\usepackage[left=3cm,right=3cm,
    top=3cm,bottom=3cm,bindingoffset=0cm]{geometry}

\usepackage{graphicx}
\usepackage{color}
\usepackage{hyperref}

\usepackage{setspace}
\usepackage{indentfirst}
\usepackage{textcomp}
\usepackage{ifthen}
\usepackage{calc}

\title{Теория Вероятностей и Математическая Статистика\\
ФИИТ, 2 курс, 4 семестр}
\author{Практика 2}

\begin{document}
\maketitle

\section{Начало, вопросы по предыдущей лекции}

\subsection{Бутерброд с маслом}

Рассмотрим опыт с падающим бутербродом с маслом. Элементарных исходов в нем два: упало маслом или хлебом.

По жизненному опыту масло падает на поверхность чаще, чем хлебом. Однако, говорить, что данные исходы не равновозможны -- нельзя. Данный эксперимент чрезмерно трудно воспроизвести, в дело вступает физика.

Так что нашу КСТВ сломать этим не получается. Вообще, тут в дело вступает Оценка вероятности редких событий, до чего мы дойдем не скоро.

\subsection{Ранвовероятность != равновозможность}

Ранвовероятные бывают случайные события, так как к ним применимо понятие вероятности.

У исходов вероятность как таковую мы не рассматриваем, мы говорим, что они равновозможны и ни одна из них не предпочтительней.

Так что это не одинаковые понятия, важно помнить об этом.

\section{Задача 1}

В чулане 10 разных пар ботинок. Из них наугад берут 4 ботинка. Найти вероятность того, что среди выбранных ботинков найдется пара.

\quad

В рамках КСТВ нам нужно знать общее колво исходов $N$, сам исход $A$ и колво исходов $N_A$, благоприятствующих $A$. Отсюда находим вероятность $\frac{N_A}{N}$.

Мы вибираем ботинки, значит по сути заполняем следующую коробку:
\begin{center}
  [\quad|\quad|\quad|\quad]
\end{center}

Тогда всего событий будет $N = 20 \cdot 19 \cdot 18 \cdot 17$.

Далее поймем, а какое событие $A$ нам нужно. Если у нас есть 4 ботинка, тогда у нас может быть $0$ пар, $1$ пара, $2$ пары. Значит, можем выделить следующее событие:

\begin{center}
  $A$ = $\{$ Найдется пара $\} = \{$ 1 пара, 2 пары$\}$

  $\overline{A}$ = $\{$ нет пар $\} = \{$ 0 пар$\}$
\end{center}

Второе событие найти проще, так как меньше элементарных исходов. Для нахождения, пронумеруем пары ботинок:

\quad

1 : 20, 19

2 : 18, 17

...

10 : 2,  1

\quad

И теперь проведем вычисления:

$$N_{\overline{A}} = 20 \cdot 18 \cdot 16 \cdot 14$$

$$P(\overline{A}) = \frac{N_{\overline{A}}}{N} = \frac{20 \cdot 18 \cdot 16 \cdot 14}{20 \cdot 19 \cdot 18 \cdot 17}$$

$$P(A) = 1 - P(\overline{A}) = 1 - \frac{16 \cdot 14}{19 \cdot 17} = \ldots $$

\quad

(Опять же -- нахождение числа не является главной целью. Важно решение.)

\section{Задача 2}

В лифт девяти этажного дома вошли трое. Найдите вероятность того, что для их обслуживания лифт сделает две остановки. (И гарантированно никто не выйдет тут же из лифта на первый этаж.)

\quad

В рамках КСТВ общее число вариантов $N = 64 \cdot 8 = 8^3$, а кол-во благоприятных вариантов $N_A = 7$.

Исходы такого опыта, отделениями которой являются разделения по этажам, опять же можно изобразить в виде "коробок":

\begin{center}
  [\qquad|\qquad|\qquad]

     8 \quad 8 \quad8
\end{center}

Для опреедления двух людей, которые выйдут на одном этаже, мы можем две ячейки связать вместе, принимая за одну. И, казалось бы, получим:
\begin{center}
  [\qquad\qquad|\qquad]

     8 \qquad\quad 7
\end{center}

Получаем $7 \cdot 8$. Однако какие ячейки мы можем принять за одну? Вариантов у нас несколько. Потому это количество стоит еще домножить на $C_3^2$ (именно для перебора "склеек ячеек"! В обычном переборе сочетания не участвуют): $7 \cdot 8 \cdot C_3^2$.

\quad

Однако есть и другой, более простой способ. Перечислим исходы:

\quad

3 человека на этаже

2 на этаже

1 на этаже

\quad

Наше событие $A = \{$ 2 человека на этаже $\}$.

Тогда $\overline{A} = \{$ 3 на этаже, 1 на этаже $\}$.

$$N_{\overline{A}} = 8 + 8 \cdot 7 \cdot 6 = 8 \cdot 48 $$

Отсюда легко найти вероятность $P(A)$.

\section{Задача 3}

Шары: 3 белых, 5 черных. Наугад располагаются в ряд. С какой веротностью белые шары в этом ряду будут лежать подряд?

\quad

Пронумеруем шары.

$$N = 8!$$

$$N_A = 6 \cdot 3! \cdot 5!$$

(кто может -- объясните)

$$P(A) = \frac{N}{N_A} = \frac{6 \cdot 3! \cdot 5!}{8!} = \frac{6 \cdot 6}{8 \cdot 7 \cdot 6} = \frac{6}{8 \cdot 7}$$

\quad

Можно предложить еще один вариант решения. 

Пусть шары не имеют номеров (необычно, но все же). То есть отличаются цветом и только цветом.

Общее кол-во мест для 3 белых шаров $N = C_8^3 = C_8^5$.

Благоприятных исходов: $N_A = 6$.

Итого получаем:

$$P(A) = \frac{6}{C_8^3}$$

\section{Задача 4}

Имеются карточки с числами $1, 2, \ldots, 100$. Наугад берут две карточки. Найти вероятность того, что:

а) одно из взятых чисел меньше 50, а второе больше 59;

б) одно из чисел меньше 40, а второе больше 60;

в) одно из чисел меньше 40, а второе меньше 60.

\quad

\subsection{Пункт a)}

$N = 100 \cdot 99$

$N_A = 49 \cdot 50 \cdot 2$ (умножение на 2, так как можно взять в другом порядке)

Итого $P(A) = \frac{49}{99}$

\quad

Вообще, выбор лучше делать следующим образом. Имеем:



\begin{center}
\quad1-49 \quad-- 49 шт.

50 \qquad-- 1 шт.

\qquad51-100 \quad-- 50 шт.
\end{center}


Тогда нашему событию $A$ соответствует (1, 0, 1)

$$P(A) = \frac{C_{49}^1 \cdot C_1^0 \cdot C_{50}^1}{C_{100}^1} = \frac{49 \cdot 50}{\frac{100 \cdot 99}{2}} = \frac{49}{99}$$

\subsection{Пункт б)}

Имеем следующие варианты:

\begin{center}
$39 (<40)$

\qquad\quad$21 (>= 40, <= 60)$

$40 (>60)$
\end{center}

Из этих вариантов $\rightarrow^2 (1, 0, 1)$. (Все еще вспоминаем про $2$, так как всегда можем карты взять в другом порядке.) Итого:

$$P(A) = \frac{C_{39}^1 \cdot C_21^0 \cdot C_{40}^1}{C_{100}^2} = \frac{21 \cdot 40}{\frac{100 \cdot 99}{2}} = \frac{52}{165}$$

\subsection{Пункт в)}

$$(39, 20, 41) \rightarrow^2(2, 0, 0) + (1, 1, 0)$$

$$P(A) = \frac{C_{39}^2 + C_{39}^1 \cdot C_{20}^1}{C_{100}^2} = \ldots$$

\end{document}
